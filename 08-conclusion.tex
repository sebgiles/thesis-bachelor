\chapter{Conclusion}
\label{chap:conclusion}
The models developed in this work constitute a powerful toolset for the evaluation of vehicle control strategies for traction control.
An intuitive but superficial validation of the model was made by using grahical tools.

A highly integrated simulation workflow was presented to efficiently run tests and change parameters.

The consistent modelling method used will allow easy addition of features to the simulations, such as aerodynamic effects.

The model developed in this work is quite versatile and may be used to make important design decisions concerning vehicle dynamics specifications.
\section{Future work}
\subsection{Validation}
The vehicle model still needs to be taken through a complete validation process, this will be done by comparison to the full multibody model of a Formula SAE car designed by Polimarche Racing Team using Adams View software.
\subsection{Lap simulations}
The skidpad test is easily reproducible for model validation purpouses and is also a part of the Formula Student competitions. Simulations should however be run also on different types of circuit to evaluate vehicle and controls performance in more general conditions.
Running full lap simulations requires calculation of the driver inputs, so the development of a virtual self-driving algorithm would greatly complement this work and may be an interesting starting point for a study on competitive autonomous road vehicles.
