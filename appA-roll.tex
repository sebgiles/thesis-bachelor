\chapter{Roll center approximation}
\label{chap:roll}

The SAE definition of roll center is

\textit{\say{The point in the transverse vertical plane through any pair of wheel centers at which lateral forces may be applied to the sprung mass without producing suspension roll}}

This definition directly relates the $P_w$ points to the roll center height.
Validity for chosen roll center approximation can be shown by modifying a generic equilibrium situation such as the one described in figure \todo{la stessa di prima} and applying a new side force $F'$ acting on the midpoint $M_F$ of $P_{FR}$ and $P_{FL}$, if the tyres are capable of generating the necessary friction forces ($F_1$ and $F_2$), the system will settle in a new equilibrium state.
The sum of vertical forces acting on the tyres does not change as it is equal to
$$
F_z := F_{z1}+F_{z2} = mg.
$$

As $F_z$ does not change, the increase in side friction force is entirely due to changes in the lateral friction coefficients, which are both supposed as being equal to $\mu_y$.

The fact that the spring systems are constrained in the vertical orientation means the horizontal forces acting at the ground contact points effect the chassis directly at the points $P_{FR}$ and $P_{FL}$ respectively.

The horizontal force balance is given by
$$
F + F' = F_1 + F_2 = \mu_y ( F_{z1} + F_{z2} ) = \mu_y mg
$$

To find the equilibrium roll angle the torque balance about point $M_F$ must be solved
$$
\frac{t_F}{2} \sin \phi (F_1-F_2) - \frac{t_F}{2} \cos \phi ( F_{z1} - F_{z2} ) - F (h_{CG}-q_F)  \cos \phi = 0
$$

substituting the friction forces yields
$$
\frac{t_F}{2} ( F_{z1} - F_{z2} ) [\mu_y \sin \phi - \cos \phi ] - F (h_{CG}-q_F)  \cos \phi = 0
$$

The vertical forces act through the springs (spring rate $k_F$) . As Hooke's law is linear, the difference between them is proportional to the difference between the spring lengths which is in turn obtained as a function of the roll angle.
$$
F_{z1} - F_{z2} = - k_F t_F \sin \phi
$$

The moment balance then becomes
$$
\mu_y k_F \frac{t_F^2}{2} \sin^2 \phi - k_F \frac{t_F^2}{2} \sin \phi \cos \phi - F(h_{CG}-q_F) \cos \phi = 0
$$

\todo{find source for roll angles in passenger cars being a few ° max} The equation may be linearized with respect to $\phi$.
$$
- k_F \frac{t_F^2}{2}  \phi  - F(h_{CG}-q_F) = 0
$$

Doing so makes the solution for $\phi$ independent of $\mu_y$, the only quantity influenced by $F'$. The height of point $M_F$ is then effectively the roll center height according to the SAE definition.

The roll rate $K_\phi$ is defined as roll angle per unit of lateral force applied to the CG \todo{find source}, applying this definition to the linearized equation gives the expression to be used to derive spring rates for the 6DoF model in order to match the roll-rate of the car.
$$
K_\phi = \frac{\phi}{F} = 2\frac{h_{CG}-q_F}{K_Ft_F^2}
$$
