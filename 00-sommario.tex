\sommarioITA
I motori elettrici non contengono parti mobili all'infuori dell'albero motore, questo li rende più compatti e più semplici rispetto ai motori termici di prestazioni simili. Per questo motivo l'utilizzo di più motori meno potenti per la trazione dei veicoli è fattibile senza incorrere in eccessiva complessità.
L'impiego di un motore per ogni ruota motrice può ridurre peso e dimensioni della trasmissione. Questo è dato dall'assenza dei differenziali e dal fatto che la potenza meccanica possa essere prodotta più vicino alle ruote.

Più motori possono inoltre essere controllati individualmente per produrre coppie differenti, risultando in un momento di rotazione attorno all'asse verticale, questa strategia di controllo è chiamata \textit{torque vectoring}.
Il \textit{torque vectoring} può essere usato per corregere una vettura indesideratamente sovrasterzante o sottosterzante, migliorando la sicurezza nei veicoli stradali e la maneggevolezza nelle auto da competizione.
Il controllo della trazione è inoltre ottenibile indipendentemente per ogni ruota, permettendo di percorrere curve a velocità maggiori.

L'obiettivo di questo studio è la derivazione di un modello matematico di un veicolo a quattro ruote utilizzabile in simulazione e per lo sviluppo di controllori per il torque vectoring e il controllo della trazione nei veicoli elettrici.

L'interesse per lo sviluppo del modello è sorto in vista di una prossima partecipazione alle competizioni di Formula SAE Electric da parte dell'Università Politecnica delle Marche. Il modello è quindi concepito per applicazioni competitive. L'alta rigidezza sospensiva, diffusa nelle vetture che competono in queste competizioni, è stata tenuta in considerazione per ottenere alcune approssimazioni.

La modellazione delle dinamiche di rollio rende prevedibili i transitori di trasferimento di carico, garantendo maggiore precisione durante le manovre ad alta frequenza come i cambi di corsia, le piccole chicane e gli stretti slalom che tipicamente compongono i percorsi degli eventi di Autocross ed Endurance in Formula SAE.
L'inclusione del comportamento di beccheggio e rollio permette anche di studiare la correlazione tra i trasferimenti di carico e l'escursione delle sospensioni, essendo quest'ultima una grandezza fisica facilmente misurabile in tempo reale e quindi utilizzabile come ingresso al sistema di controllo del veicolo.

Il capitolo \ref{chap:6dof} descrive fisicamente un modello a sei gradi di libertà, successivamente ne viene fornita la formulazione lagrangiana. Nel capitolo \ref{chap:12dof} il modello viene esteso con 6 gradi di libertà aggiuntivi per la descrizione della rotazione delle ruote e delle dinamiche di sterzo anteriore e posteriore.

La dinamica di sterzo per le ruote posteriori è stata inclusa nella seconda versione del modello, quest permetterà di valutare eventuali benefici portati dall'utilizzo di quattro ruote sterzanti, tecnica esplicitamente consentita dal regolamento Formula SAE.

La formulazione lagrangiana è stata costruita simbolicamente utilizzando la  MATLAB Symbolic Math Toolbox per facilitare il calcolo di derivate e trasformazioni di coordinate. Tutti gli script sono scritti in forma vettoriale così da rendere il codice più compatto e leggibile.

Sono state scritte delle funzioni per risolvere algebricamente le equazioni di Lagrange e fornire una rappresentazione in spazio di stato del sistema.

Le simulazioni sono state eseguite in ambiente Simulink, in cui i due modelli matematici ottenuti sono stati implementati in forma modulare, accompagnati da un modello empirico per le forze di attrito generate dagli pneumatici.

La produzione di animazioni bidimensionali della vettura ha facilitato la risoluzione degli errori di programmazione rendendo più intuibile il siginificato dei risultati numerici.
