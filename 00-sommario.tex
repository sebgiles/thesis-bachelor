\sommarioITA
I motori elettrici non contengono parti mobili all'infuori dell'assieme dell'albero motore, questo li rende più compatti e più semplici rispetto a motori termici di prestazioni simili. L'utilizzo di più motori per la trazione dei veicoli è quindi fattibile senza incorrere in eccessivo peso o complessità. L'Utilizzo di un motore per ogni ruota motrice può ridurre peso, dimensioni e complessità della trasmissione. Questo è dato dall'assenza dei differenziali e dal fatto che la potenza meccanica possa essere prodotta più vicino alle ruote.

I motori possono inoltre essere controllati individualmente per produrre coppie differenti, risultando in un momento di rotazione attorno all'asse verticale, questa strategia di controllo è chiamata torque vectoring.
Il torque vectoring può essere usato per corregere una vettura sovrasterzante o sottsterzante, migliorando la sicurezza dei veicoli stradali e la maneggevolezza di auto da competizione.
Il controllo della trazione è inoltre ottenibile indipendentemente per ogni ruota, permettendo di percorrere curve a velocità maggiori.

L'obiettivo di questo studio è la derivazione di un modello matematico utilizzabile in simulazione e per lo sviluppo di controllori per il torque vectoring e il controllo della trazione nei veicoli elettrici.

L'interesse per lo sviluppo del modello è sorto in vista di una prossima partecipazione alle competizioni di Formula Student Electric. Il modello è quindi concepito per applicazioni competitive. L'alta rigidezza sospensiva, diffusa nelle vetture che competono in queste competizioni, è stata tenuta in considerazione per ottenere alcune approssimazioni.

La modellazione delle dinamiche di rollio rende prevedibili i transitori di trasferimento di carico, garantendo maggiore precisione durante le manovre a frequenza più alta come i cambi di corsia, le piccole chicane e gli stretti slalom che tipicamente compongono i percorsi degli eventi di Autocross ed Endurance in Formula Student.
L'inclusione del comportamento di beccheggio e rollio permette anche di studiare la correlazione tra i trasferimenti di carico e l'escursione delle sospensioni, essendo quest'ultima una grandezza fisica facilmente misurabile in tempo reale e quindi utilizzabile come ingresso al sistema di controllo del veicolo.
Le dinamiche di sterzo posteriori sono incluse nel secondo modello presentato poichè il regolamento Formula Student permette esplicitamente l'utilizzo di quattro ruote sterzanti e la modellazione non ha richiesto alcun impegno aggiuntivo.

Il capitolo \ref{chap:6dof} descrive fisicamente un modello a sei gradi di libertà, successivamente ne viene fornita la formulazione lagrangiana. Nel capitolo \ref{chap:12dof} il modello viene esteso con 6 gradi di libertà aggiuntivi per la descrizione della rotazione delle ruote e delle dinamiche di sterzo anteriore e posteriore.

La formulazione lagrangiana è stata costruita simbolicamente utilizzando la  MATLAB Symbolic Math Toolbox per facilitare le derivate e le trasformazioni di coordinate.
Sono state scritte delle funzioni per risolvere algebricamente le equazioni di Lagrange e fornire una rappresentazione in spazio di stato del sistema.
Tutti gli script sono scritti in forma vettoriale rendendo il codice più compatto e leggibile
Le simulazioni sono state eseguite in ambiente simulink, in cui i due modelli matematici ottenuti sono stati implementati in forma modulare, accompagnati da un modello empirico per le forze di attrito generate dagli pneumatici.
L'utilizzo di un'animazione 2D della vettura ha facilitato la risoluzione dei problemi rendendo più intuibile il siginificato dei risultati numerici.
