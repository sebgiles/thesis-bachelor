\chapter{Six degree-of-freedom vehicle model}
\label{chap:6dof}
\section{6DoF Model structure}
\label{sec:6dofconcept}
The state of the car is identified by the six degrees of freedom required to define the spacial position and orientation of its chassis. The car interacts with the road through a simplified suspension system.
The model outputs are the vertical forces on each of the four wheels, obtained as the forces acting through each of the suspesion systems, and the velocities of the respective ground contact points, all of which may be used to calculate friction by an external tyre model.
Inputs to the model are the longitudinal and lateral forces acting on each wheel and the angle of the front wheels which is used to rotate the force vectors acting on them, in order to obtain steering control.
The block diagram in \todo{figura} shows the intended flow of information for this model working in conjuction with a tyre model.
\section{Chassis description}
\label{sec:body}
The road is assumed to be a horizontal plane, so we can define the $xyz$ inertial reference system where $z$ is pointed downwards, in the direction of gravity, $x$ and $y$ lie on the road surface and may be arbitrarily chosen as long as the right-handedness of the reference frame is guaranteed.
The chassis is represented by a rigid body whose mass $m$ and rotational inertia matrix $I$ resemble those of the entire vehicle, inclusive of the driver and the wheels.
The body reference system $x'y'z'$ is fixed to the chassis and originates at the Center of Gravity. An unambiguous definition of the axes is made in the static equilibrium orientation when the suspension system is bearing the weight of the car and no other forces are at play. In this condition $z'$ is defined to be parallel to $z$ and likewise pointed downwards, $x'$ is directed forwards, parallel to the road plane, and the $y'$ axis is oriented to the right of the car as seen by the driver.
The attitude of the chassis is defined by the Tait-Bryan angles mapping the inertial system axes to the chassis axes \todo{appendix}
\todo{figura SAE J670}. Note that the given definition for the chassis reference system implies zero roll and pitch at static equilibrium.
The chassis is assumed to be symmetrical about the $x'z'$ plane, this identifies the $y'$ axis as a principle axis of inertia and eliminates all the related product of inertia terms, yielding
$$ I = \begin{bmatrix}
    I_{xx} & 0      & I_{xz}\\
    0      & I_{yy} & 0     \\
    I_{xz} & 0      & x_{zz}
\end{bmatrix}.$$

\todo{usare molteplici figure: 1. sistemi di riferimento e coordinate lagrangiane - 2. sospensioni su un solo asse - 3. vista 3d delle sospensioni su una sola ruota - 4. vista 3d completa ma senza molle}


\section{Suspension}
\label{sec:suspension}
Suspension is represented by four vertical spring-damper systems associated with each wheel of the car. Quantities relating to each of the systems will be distinguished by subscripts according to Table \ref{table:subscripts}, the $w$ subscript is used when referring generically to any one of them.

\begin{table}[ht]
\caption{Wheel and suspension systems subscripts} % title of Table
\centering % used for centering table
\begin{tabular}{l l l l} % left aligned columns (4 columns)
\hline\hline %inserts double horizontal lines
Subscript & Pertaining wheel or suspension system \\ [0.5ex] % inserts table
%heading
\hline % inserts single horizontal line
$FR$ & Front Right \\ % inserting body of the table
$FL$ & Front Left \\
$RR$ & Rear Right \\
$RL$ & Rear Left \\ [1ex] % [1ex] adds vertical space
\hline %inserts single line
\end{tabular}
\label{table:subscripts} % is used to refer this table in the text
\end{table}

The upper end of each suspension system is attached to an appropriate point $P_w$ statically defined with respect to the chassis, the lower end point, $W_w$, is obtained by projecting $P_w$ onto the road plane.

The correct weight transfer effects are obtained by assigning the corresponding wheel contact point $x$ and $y$ coordinates to each of the $P_w$ points, these are easily calculated from the front and rear track lengths ($t_f$ and $t_r$) and the distances of the front and rear axles from the CoG.

The height of the points $P_w$ are finally chosen in order to obtain realistic roll dynamics.

All automotive suspension systems may be characterized by a roll center for each axle. This is the point at which lateral forces acting on the wheels are reacted to the chassis.
The roll center is typically below the center of mass, causing vehicles to lean towards the outside of turns when driving, this is in line with the situation shown in \todo{image}, where the lateral force at the center of mass is the centrifugal force acting in the non inertial reference frame of the car, balanced by friction acting on the wheel contact points.
The roll center height is determined by suspension kinematics and dynamic load distribution so it is not always constant. The SAE definition of roll center is

\textit{\say{The point in the transverse vertical plane through any pair of wheel centers at which lateral forces may be applied to the sprung mass without producing suspension roll}}

This definition directly relates the $P_w$ points to the roll center height.

This can be explained by modifying a generic equilibrium situation such as the one described in figure \todo{la stessa di prima} and applying a new side force $F'$ acting on the midpoint $M_F$ of $P_{FR}$ and $P_{FL}$, if the tyres are capable of generating the necessary friction forces ($F_1$ and $F_2$), the system will settle in a new equilibrium state.
The sum of vertical forces acting on the tyres does not change as it is equal to
$$
F_z := F_{z1}+F_{z2} = mg.
$$

As $F_z$ does not change, the increase in side friction force is entirely due to changes in the lateral friction coefficients, which are both taken as equal to $\mu_y$.

The fact that the spring systems are constrained in the vertical orientation means the horizontal forces acting at the ground contact points effect the chassis directly at the points $P_{FR}$ and $P_{FL}$ respectively.

The horizontal force balance is given by
$$
F + F' = F_1 + F_2 = \mu_y ( F_{z1} + F_{z2} ) = \mu_y mg
$$

To find the equilibrium roll angle the torque balance about point $M_F$ must be solved

$$
\frac{t_F}{2} \sin \phi (F_1-F_2) - \frac{t_F}{2} \cos \phi ( F_{z1} - F_{z2} ) - F h_F  \cos \phi = 0
$$
substituting the friction forces yields
$$
\frac{t_F}{2} ( F_{z1} - F_{z2} ) [\mu_y \sin \phi - \cos \phi ] - F h_F  \cos \phi = 0
$$

The vertical forces act through the suspension springs. As Hooke's law is linear, the difference between them is proportional to the difference between the spring lengths which is in turn obtained as a function of the roll angle.
$$
F_{z1} - F_{z2} = - k_F t_F \sin \phi
$$

The moment balance then becomes

$$
\mu_y k_F \frac{t_F^2}{2} \sin^2 \phi - k_F \frac{t_F^2}{2} \sin \phi \cos \phi - Fh_F \cos \phi = 0
$$

\todo{find source for roll angles in passenger cars being a few ° max} The equation may be linearized with respect to $\phi$.

$$
- k_F \frac{t_F^2}{2}  \phi  - Fh_F = 0
$$

Doing so makes the solution for $\phi$ independent of $\mu_y$, the only quantity influenced by $F'$. The height of point $M_F$ is then effectively the roll center height according to the SAE definition.

The roll rate $K_\phi$ is defined as roll angle per unit of lateral force applied to the CG \todo{find source}, applying this definition to the linearized equation gives

$K_\phi = \frac{\phi}{F} = \frac{2h}{K_Ft_F^2}$

This expression can be used to derive spring rates for the 6DoF model in order to match the roll-rate of the car.

\begin{table}[ht]
\caption{Coordinates of the suspension force application points} % title of Table
\centering % used for centering table
\begin{tabular}{l l l l l} % left aligned columns (4 columns)
\hline\hline %inserts double horizontal lines
Chassis system co-ordinate & $P_FR$ & $P_FL$ & $P_RR$ & $P_RL$ \\ [0.5ex] % inserts table
%heading
\hline % inserts single horizontal line
$ x$ & $ l_f$ & $ l_f$ & $-l_r $ & $-l_r $\\ % inserting body of the table
$ y$ & $ \frac{t_f}{2} $ & $ -\frac{t_f}{2}$ & $ \frac{t_r}{2}$ & $ -\frac{t_r}{2}$\\ % inserting body of the table
$ z$ & $ h_{CG} - q_F $& $ h_{CG} - q_F $ & $ h_{CG} - q_R$ & $ h_{CG} - q_R$ \\ [1ex] % [1ex] adds vertical space
\hline %inserts single line
\end{tabular}
\label{table:susppoints} % is used to refer this table in the text
\end{table}

\section{Limitations}
\label{sec:6doflimits}
While this model benefits from the simplicity of it's description
The main limitation of this model is the lack of the wheel rotational dynamics, which leads to missing reaction torques present during acceleration and braking.
\section{6DoF Dynamics equation set}
\label{sec:6dofeq}
\todo{districare il groviglio che mi lascia Matlab}
