\chapter{Six degree-of-freedom vehicle model}
\label{chap:6dof}
\section{6DoF Model structure}
\label{sec:6dofconcept}
The state of the car is identified by the six degrees of freedom required to define the position and orientation of its chassis. The car interacts with the road through a simplified suspension system.
The model outputs are the vertical forces on each of the four wheels and the velocities of the respective ground contact points, which may be used to calculate friction by an external tyre model.
Inputs to the model are the longitudinal and lateral forces acting on each wheel and the steering angle. The latter effectively modifies the slip angle and slip ratio of the front wheels and rotates the consequent force vectors thereby allowing direction control.
The block diagram in figure \ref{6flow} shows the intended flow of information for this model working in conjuction with a tyre model.
\begin{figure}[ht]
  \centering
  \includegraphics[width=\textwidth]{images/6flow.png}
  \caption{Information flow between the tyre model and the 6dof vehicle model.}
  \label{6flow}
\end{figure}
\section{Chassis description}
\label{sec:body}
The road is assumed to be a horizontal plane. The $xyz$ inertial reference system is defined by $z$ pointed downwards, in the direction of gravity, whilst $x$ and $y$ lie on the road surface and may be arbitrarily chosen as long as the right-handedness of the reference frame is guaranteed.

The chassis is represented by a rigid body whose mass $m$ and rotational inertia matrix $I$ resemble those of the entire vehicle, inclusive of the driver and the wheels.

The body reference system $x'y'z'$ is fixed to the chassis and originates at the center of mass. An unambiguous definition of the axes is made in the static equilibrium orientation when the suspension system is bearing the weight of the car and no other forces are at play (this will be later referred to as the \textit{resting condition}). In this condition $z'$ is defined to be parallel to $z$ and likewise pointed downwards, $x'$ is directed forwards, parallel to the road plane, and the $y'$ axis is oriented to the right of the car as seen by the driver. The chassis reference system here defined is in line with the SAE J670\cite{j670} definition depicted in figure \ref{saeaxes}.
\begin{figure}[ht]
  \centering
  \includegraphics[height = 7cm]{images/saeaxes}
  \caption{The chassis reference system as defined in the SAE J670 standard.}
  \label{saeaxes}
\end{figure}

The attitude of the chassis is defined via the Tait-Bryan angles mapping the inertial system axes to the chassis axes: yaw ($\psi$), pitch ($\theta$) and roll ($\phi$).  Note that the previously given definition for the chassis reference system implies zero roll and pitch at in the resting condition

An undercarriage reference system is also defined as having origin in the CoG projection to the road, its axes $x_uy_uz_u$ are obtained by applying only the yaw rotation to the inertial reference system.

The $R_\psi$ and $R$ matrices can be used to calculate inertial system co-ordinates from undercarriage system or vehicle system co-ordinates respectively.

$$
R_\psi = \begin{bmatrix}
cos(\psi) & -sin(\psi)      & 0\\
sin(\psi) & cos(\psi) & 0     \\
0 & 0      & 1
\end{bmatrix}.
$$

$$
R = \begin{bmatrix}
cos(\psi) & -sin(\psi)      & 0\\
sin(\psi) & cos(\psi) & 0     \\
0 & 0      & 1
\end{bmatrix}.\begin{bmatrix}
cos(\theta) & 0      & sin(\theta) \\
0 & 1 & 0  \\
-sin(\theta) & 0 & cos(\theta)
\end{bmatrix}.\begin{bmatrix}
cos(\phi) & -sin(\phi)      & 0\\
sin(\phi) & cos(\phi) & 0     \\
0 & 0      & 1
\end{bmatrix}.
$$

Geometric point coordinates can be brought into the inertial frame by rotating the point co-ordinates using the approriate rotation matrix and adding the original reference system origin co-ordinates.

The chassis is assumed to be symmetrical about the $x'z'$ plane, this identifies the $y'$ axis as a principle axis of inertia and eliminates all the related product of inertia terms from the inertia tensor evaluated in the chassis reference system.
$$
I = \begin{bmatrix}
I_{xx} & 0      & I_{xz}\\
0      & I_{yy} & 0     \\
I_{xz} & 0      & I_{zz}
\end{bmatrix}.
$$

Quantities relating to each of the four wheels or the correpsonding suspension systems will be distinguished by subscripts according to Table \ref{table:subscripts}, the $w$ subscript is used when referring generically to any one of them.

\begin{table}[ht]
  \caption{Wheel and suspension systems subscripts} % title of Table
  \centering % used for centering table
  \begin{tabular}{l l l l} % left aligned columns (4 columns)
    \hline\hline %inserts double horizontal lines
    Subscript & Pertaining wheel or suspension system \\ [0.5ex] % inserts table
    %heading
    \hline % inserts single horizontal line
    $FR$ & Front Right \\ % inserting body of the table
    $FL$ & Front Left \\
    $RR$ & Rear Right \\
    $RL$ & Rear Left \\ [1ex] % [1ex] adds vertical space
    \hline %inserts single line
  \end{tabular}
  \label{table:subscripts} % is used to refer this table in the text
\end{table}

\section{Suspension}
\label{sec:suspension}
Suspension is represented by four vertical linear spring-damper systems associated with each wheel of the car.

The upper end of each suspension system is attached to an appropriate point $P_w$ statically defined with respect to the chassis, the lower end point is obtained by projecting $P_w$ onto the road plane.
The spring length $h_w$ is given by the distance between $P_w$ and the road plane, so it is equal to the opposite of the inertial system $z$ coordinate of $P_w$.

\begin{figure}[ht]
  \centering
  \includegraphics[width=\textwidth]{images/3dview}
  \caption{Description of the 6 DoF model geometry.}
  \label{3dview}
\end{figure}

The correct weight transfer effects are obtained by assigning the corresponding wheel contact point $x$ and $y$ coordinates to each of the $P_w$ points. These are easily calculated from the front and rear track lengths ($t_F$ and $t_R$) and the distances of the front and rear axles from the CoG ($l_F$ and $l_R$).

The $z$ co-ordinates of the $P_w$ points with respect to the chassis system are finally chosen in order to obtain realistic roll dynamics.

Automotive suspension systems can be characterized by a roll center for each axle. This is the point at which lateral forces acting on the wheels are reacted to the chassis.
The roll center is usually below the center of mass, causing vehicles to lean towards the outside of turns when driving, this is in line with the situation shown in figure \ref{roll}, where the lateral force at the center of mass is the centrifugal force acting in the non inertial reference frame of the car, balanced by friction acting on the wheel contact points.

\begin{figure}[ht]
  \centering
  \includegraphics[height = 7cm]{images/roll}
  \caption{Body roll during cornering.}
  \label{roll}
\end{figure}

The roll center height, as defined by the SAE in J670, is determined by suspension kinematics and dynamic load distribution so it is almost never constant.

The roll center heights ($q_F$ and $q_R$) are a common design specification in race cars.
Their values can be obtained by means of multibody simulation, experimental tests or kinematic approximations \cite{gillespie}.
Usually $q_F$ and $q_R$ are given with respect to ground in the resting condition even though they are more relevant when considered with respect to the CoG height and tend to move with the chassis of the vehicle.

In the developed model, each roll center is approximated by the midpoint of the left and right $P_w$ points of the corresponding axle. This approximation is justified in Appendix \ref{chap:roll}.
When the model is in the resting condition the height of each of the $P_w$ points is made to match the corresponding roll center height by completing their definition as in table \ref{table:susppoints}.

\begin{table}[ht]
  \caption{Coordinates of the suspension force application points} % title of Table
  \centering % used for centering table
  \begin{tabular}{l l l l l} % left aligned columns (4 columns)
    \hline\hline %inserts double horizontal lines
    Chassis system co-ordinate & $P_{FR}$ & $P_{FL}$ & $P_{RR}$ & $P_{RL}$ \\ [0.5ex] % inserts table
    %heading
    \hline % inserts single horizontal line
    $ x$ & $ l_f$ & $ l_f$ & $-l_r $ & $-l_r $\\ % inserting body of the table
    $ y$ & $ \frac{t_f}{2} $ & $ -\frac{t_f}{2}$ & $ \frac{t_r}{2}$ & $ -\frac{t_r}{2}$\\ % inserting body of the table
    $ z$ & $ h_{CG} - q_F $& $ h_{CG} - q_F $ & $ h_{CG} - q_R$ & $ h_{CG} - q_R$ \\ [1ex] % [1ex] adds vertical space
    \hline %inserts single line
  \end{tabular}
  \label{table:susppoints} % is used to refer this table in the text
\end{table}

In appendix \ref{chap:roll} it is also shown that the linear spring rates for the model should be calculated from the roll rates of the vehicle, using the formulas
$$
k_F = \frac{\phi}{F} = 2\frac{h_{CG}-q_F}{K_{\phi F} t_F^2}
$$
$$
k_R = \frac{\phi}{F} = 2\frac{h_{CG}-q_R}{K_{\phi R} t_F^2}.
$$


To make the the car sit in the resting position with zero pitch angle and the correct CoG height the unloaded spring lengths must be
$$ h_{F0} = q_F + \frac{mg}{2k_F}\frac{l_R}{l_R+l_F} $$
$$ h_{R0} = q_r + \frac{mg}{2k_R}\frac{l_F}{l_R+l_F} $$

\section{Lagrangian formulation}
\label{sec:6doflag}
Equations of motion were obtained by solving the lagrangian formulation of the system with respect to the inertial reference frame.
The chosen lagrangian coordinates are, in order, the yaw, pitch and roll angles and the coordinates of the chassis CoG
$$
q = \begin{bmatrix}
\psi \\
\theta \\
\phi \\
x_{CG} \\
y_{CG} \\
z_{CG}
\end{bmatrix}.
$$
The gravitational potential energy is
$$U_{grav} = -mgz_{CG}$$
The potential energy stored in the suspension springs is
$$ U_{spring} = \frac{1}{2} k_F [(h_{FR} - h_{F0})^2 + (h_{FL} - h_{F0})^2] +  \frac{1}{2} k_R [(h_{RR} - h_{R0})^2 + (h_{RL} - h_{R0})^2] $$

The total potential energy function is
$$ U = U_{spring} + U_{grav} $$

The damping effect of the shock absorbers is modelled by the Rayleigh dissipation function
$$ D = \frac{1}{2} b_F (\dot h_{FR}^2 + \dot h_{FL}^2) + \frac{1}{2} b_R (\dot h_{RR}^2 + \dot h_{RL}^2) $$
$b_F$ and $b_r$ are the effective damping coefficients for vertical movements.

The translational kinetic energy of the car is given by
$$ T_{trans} = \frac{1}{2} m (\dot x_{CG}^2 +\dot y_{CG}^2 +\dot z_{CG}^2 ) $$

$E$ is the mapping between the Tait-Bryan angle derivatives and the angular velocity vector
$$
E=\begin{bmatrix}
0 & -sin(\psi)      & cos(\theta)cos(\psi)\\
0      & cos(\psi) &  cos(\theta)sin(\psi) \\
1 & 0      & -sin(\theta)
\end{bmatrix}.
$$

The angular velocity vector with respect to the inertial frame is obtained from the roll, pitch and yaw rates as
$$
\omega' = E\begin{bmatrix}
\dot\psi \\
\dot\theta \\
\dot\phi
\end{bmatrix}
$$

Computing the kinetic energy requires the angular velocity vector $\omega$ to be expressed with respect to the same axes as the inertia matrix, i.e. the chassis reference frame. This is easily done by
$$\omega = R^{-1}\omega'$$

The rotational kinetic energy is
$$ T_{rot} = \frac{1}{2}\omega^T I \omega $$

The total kinetic energy function is then
$$ T = T_{rot} + T_{trans} $$

The steering angle input $\delta$ acts like a time varying parameter in the model, influencing the wheel system rotation matrices.
The tyre horizontal force component inputs $F_{wx}$ $F_{wy}$ are to be expressed with respect to each of the wheel reference systems.

To obtain the generalized forces vector $Q$ the external forces expressed in vector form in the wheel frames are rotated into inertial frame co-ordinates and applied to the $P_w$ points.

\section{6DoF Dynamics equation set}
\label{sec:6dofeq}
The unsolved Lagrange Equations may be expressed in the form

$$M(\theta,\phi)\ddot q = f(q, \dot q) + Q(q,\delta)u$$

$$
M(\phi,\theta)=\left(\begin{array}{cccccc} M_{\mathrm{\psi},\mathrm{\psi}} & M_{\mathrm{\psi},\mathrm{\theta}} & M_{\mathrm{\psi},\mathrm{\phi}} & 0 & 0 & 0\\ M_{\mathrm{\psi},\mathrm{\theta}} & M_{\mathrm{\theta},\mathrm{\theta}} & M_{\mathrm{\theta},\mathrm{\phi}} & 0 & 0 & 0\\ M_{\mathrm{\psi},\mathrm{\phi}} & M_{\mathrm{\theta},\mathrm{\phi}} & M_{\mathrm{\phi},\mathrm{\phi}} & 0 & 0 & 0\\ 0 & 0 & 0 & m & 0 & 0\\ 0 & 0 & 0 & 0 & m & 0\\ 0 & 0 & 0 & 0 & 0 & m \end{array}\right)
$$

\begin{align*}
  M_{\mathrm{\psi},\mathrm{\psi}}&=\mathrm{I_{zz}}\,{\cos\left(\mathrm{\theta}\right)}^2\,{\cos\left(\mathrm{\phi}\right)}^2+\mathrm{I_{yy}}\,{\cos\left(\mathrm{\theta}\right)}^2\,{\sin\left(\mathrm{\phi}\right)}^2-2\,\mathrm{I_{xz}}\,\cos\left(\mathrm{\theta}\right)\,\cos\left(\mathrm{\phi}\right)\,\sin\left(\mathrm{\theta}\right)+\mathrm{I_{xx}}\,{\sin\left(\mathrm{\theta}\right)}^2 \\
  M_{\mathrm{\psi},\mathrm{\theta}}&=\sin\left(\mathrm{\phi}\right)\,\left(\mathrm{I_{xz}}\,\sin\left(\mathrm{\theta}\right)+\mathrm{I_{yy}}\,\cos\left(\mathrm{\theta}\right)\,\cos\left(\mathrm{\phi}\right)-\mathrm{I_{zz}}\,\cos\left(\mathrm{\theta}\right)\,\cos\left(\mathrm{\phi}\right)\right) \\
  M_{\mathrm{\psi},\mathrm{\phi}}&=\mathrm{I_{xz}}\,\cos\left(\mathrm{\theta}\right)\,\cos\left(\mathrm{\phi}\right)-\mathrm{I_{xx}}\,\sin\left(\mathrm{\theta}\right) \\
  M_{\mathrm{\theta},\mathrm{\theta}}&=\mathrm{I_{yy}}-\mathrm{I_{yy}}\,{\sin\left(\mathrm{\phi}\right)}^2+\mathrm{I_{zz}}\,{\sin\left(\mathrm{\phi}\right)}^2 \\
  M_{\mathrm{\theta},\mathrm{\phi}}&=-\mathrm{I_{xz}}\,\sin\left(\mathrm{\phi}\right) \\
  M_{\mathrm{\phi},\mathrm{\phi}}&=\mathrm{I_{xx}}
\end{align*}

$$
f = \begin{bmatrix}
f_1 \\
f_2 \\
f_3 \\
0 \\
0 \\
f_6
\end{bmatrix}.
$$
\begin{changemargin}{0cm}{5cm}
$f_1 = \mathrm{\dot \theta}\,\mathrm{\dot \phi}\,\mathrm{Ixx}\,\cos\left(\mathrm{\theta}\right)-{\mathrm{\dot \theta}}^2\,\mathrm{Ixz}\,\cos\left(\mathrm{\theta}\right)\,\sin\left(\mathrm{\phi}\right)+{\mathrm{\dot \phi}}^2\,\mathrm{Ixz}\,\cos\left(\mathrm{\theta}\right)\,\sin\left(\mathrm{\phi}\right)-\mathrm{\dot \theta}\,\mathrm{\dot \psi}\,\mathrm{Ixx}\,\sin\left(2\,\mathrm{\theta}\right)-\mathrm{\dot \theta}\,\mathrm{\dot \phi}\,\mathrm{Iyy}\,\cos\left(\mathrm{\theta}\right)\,{\cos\left(\mathrm{\phi}\right)}^2+\mathrm{\dot \theta}\,\mathrm{\dot \phi}\,\mathrm{Izz}\,\cos\left(\mathrm{\theta}\right)\,{\cos\left(\mathrm{\phi}\right)}^2+2\,\mathrm{\dot \theta}\,\mathrm{\dot \psi}\,\mathrm{Ixz}\,{\cos\left(\mathrm{\theta}\right)}^2\,\cos\left(\mathrm{\phi}\right)+\mathrm{\dot \theta}\,\mathrm{\dot \phi}\,\mathrm{Iyy}\,\cos\left(\mathrm{\theta}\right)\,{\sin\left(\mathrm{\phi}\right)}^2-\mathrm{\dot \theta}\,\mathrm{\dot \phi}\,\mathrm{Izz}\,\cos\left(\mathrm{\theta}\right)\,{\sin\left(\mathrm{\phi}\right)}^2-2\,\mathrm{\dot \theta}\,\mathrm{\dot \psi}\,\mathrm{Ixz}\,\cos\left(\mathrm{\phi}\right)\,{\sin\left(\mathrm{\theta}\right)}^2+{\mathrm{\dot \theta}}^2\,\mathrm{Iyy}\,\cos\left(\mathrm{\phi}\right)\,\sin\left(\mathrm{\theta}\right)\,\sin\left(\mathrm{\phi}\right)-{\mathrm{\dot \theta}}^2\,\mathrm{Izz}\,\cos\left(\mathrm{\phi}\right)\,\sin\left(\mathrm{\theta}\right)\,\sin\left(\mathrm{\phi}\right)+2\,\mathrm{\dot \theta}\,\mathrm{\dot \psi}\,\mathrm{Izz}\,\cos\left(\mathrm{\theta}\right)\,{\cos\left(\mathrm{\phi}\right)}^2\,\sin\left(\mathrm{\theta}\right)-2\,\mathrm{\dot \phi}\,\mathrm{\dot \psi}\,\mathrm{Iyy}\,{\cos\left(\mathrm{\theta}\right)}^2\,\cos\left(\mathrm{\phi}\right)\,\sin\left(\mathrm{\phi}\right)+2\,\mathrm{\dot \phi}\,\mathrm{\dot \psi}\,\mathrm{Izz}\,{\cos\left(\mathrm{\theta}\right)}^2\,\cos\left(\mathrm{\phi}\right)\,\sin\left(\mathrm{\phi}\right)+2\,\mathrm{\dot \theta}\,\mathrm{\dot \psi}\,\mathrm{Iyy}\,\cos\left(\mathrm{\theta}\right)\,\sin\left(\mathrm{\theta}\right)\,{\sin\left(\mathrm{\phi}\right)}^2-2\,\mathrm{\dot \phi}\,\mathrm{\dot \psi}\,\mathrm{Ixz}\,\cos\left(\mathrm{\theta}\right)\,\sin\left(\mathrm{\theta}\right)\,\sin\left(\mathrm{\phi}\right)$

$ f_2 = {\mathrm{\dot \phi}}^2\,\mathrm{Ixz}\,\cos\left(\mathrm{\phi}\right)+\frac{{\mathrm{\dot \psi}}^2\,\mathrm{Ixx}\,\sin\left(2\,\mathrm{\theta}\right)}{2}-k_{F}\,{l_{F}}^2\,\sin\left(2\,\mathrm{\theta}\right)-k_{R}\,{l_{R}}^2\,\sin\left(2\,\mathrm{\theta}\right)-2\,\mathrm{\dot \theta}\,b_{F}\,{l_{F}}^2\,{\cos\left(\mathrm{\theta}\right)}^2-2\,\mathrm{\dot \theta}\,b_{R}\,{l_{R}}^2\,{\cos\left(\mathrm{\theta}\right)}^2-\mathrm{\dot \phi}\,\mathrm{\dot \psi}\,\mathrm{Ixx}\,\cos\left(\mathrm{\theta}\right)+\mathrm{\dot \phi}\,\mathrm{\dot \psi}\,\mathrm{Iyy}\,\cos\left(\mathrm{\theta}\right)-\mathrm{\dot \phi}\,\mathrm{\dot \psi}\,\mathrm{Izz}\,\cos\left(\mathrm{\theta}\right)-2\,{Z_{F}}^2\,k_{F}\,\cos\left(\mathrm{\phi}\right)\,\sin\left(\mathrm{\theta}\right)-2\,{Z_{R}}^2\,k_{R}\,\cos\left(\mathrm{\phi}\right)\,\sin\left(\mathrm{\theta}\right)+2\,\mathrm{\dot z}_{\mathrm{CG}}\,b_{F}\,l_{F}\,\cos\left(\mathrm{\theta}\right)-2\,\mathrm{\dot z}_{\mathrm{CG}}\,b_{R}\,l_{R}\,\cos\left(\mathrm{\theta}\right)-{\mathrm{\dot \psi}}^2\,\mathrm{Ixz}\,{\cos\left(\mathrm{\theta}\right)}^2\,\cos\left(\mathrm{\phi}\right)-2\,Z_{F}\,k_{F}\,l_{F}\,\cos\left(\mathrm{\theta}\right)+2\,Z_{R}\,k_{R}\,l_{R}\,\cos\left(\mathrm{\theta}\right)+{\mathrm{\dot \psi}}^2\,\mathrm{Ixz}\,\cos\left(\mathrm{\phi}\right)\,{\sin\left(\mathrm{\theta}\right)}^2+2\,h_{\mathrm{CG}}\,k_{F}\,l_{F}\,\cos\left(\mathrm{\theta}\right)-2\,h_{\mathrm{CG}}\,k_{R}\,l_{R}\,\cos\left(\mathrm{\theta}\right)+\mathrm{\dot \theta}\,\mathrm{\dot \phi}\,\mathrm{Iyy}\,\sin\left(2\,\mathrm{\phi}\right)-\mathrm{\dot \theta}\,\mathrm{\dot \phi}\,\mathrm{Izz}\,\sin\left(2\,\mathrm{\phi}\right)+2\,k_{F}\,l_{F}\,z_{\mathrm{CG}}\,\cos\left(\mathrm{\theta}\right)-2\,k_{R}\,l_{R}\,z_{\mathrm{CG}}\,\cos\left(\mathrm{\theta}\right)+2\,\mathrm{\dot z}_{\mathrm{CG}}\,Z_{F}\,b_{F}\,\cos\left(\mathrm{\phi}\right)\,\sin\left(\mathrm{\theta}\right)+2\,\mathrm{\dot z}_{\mathrm{CG}}\,Z_{R}\,b_{R}\,\cos\left(\mathrm{\phi}\right)\,\sin\left(\mathrm{\theta}\right)-2\,\mathrm{\dot \theta}\,{Z_{F}}^2\,b_{F}\,{\cos\left(\mathrm{\phi}\right)}^2\,{\sin\left(\mathrm{\theta}\right)}^2-2\,\mathrm{\dot \theta}\,{Z_{R}}^2\,b_{R}\,{\cos\left(\mathrm{\phi}\right)}^2\,{\sin\left(\mathrm{\theta}\right)}^2-{\mathrm{\dot \psi}}^2\,\mathrm{Izz}\,\cos\left(\mathrm{\theta}\right)\,{\cos\left(\mathrm{\phi}\right)}^2\,\sin\left(\mathrm{\theta}\right)-2\,\mathrm{\dot \phi}\,\mathrm{\dot \psi}\,\mathrm{Iyy}\,\cos\left(\mathrm{\theta}\right)\,{\cos\left(\mathrm{\phi}\right)}^2+2\,\mathrm{\dot \phi}\,\mathrm{\dot \psi}\,\mathrm{Izz}\,\cos\left(\mathrm{\theta}\right)\,{\cos\left(\mathrm{\phi}\right)}^2-\frac{\mathrm{\dot \theta}\,{t_{F}}^2\,b_{F}\,{\sin\left(\mathrm{\theta}\right)}^2\,{\sin\left(\mathrm{\phi}\right)}^2}{2}-\frac{\mathrm{\dot \theta}\,{t_{R}}^2\,b_{R}\,{\sin\left(\mathrm{\theta}\right)}^2\,{\sin\left(\mathrm{\phi}\right)}^2}{2}+2\,Z_{F}\,h_{\mathrm{CG}}\,k_{F}\,\cos\left(\mathrm{\phi}\right)\,\sin\left(\mathrm{\theta}\right)+2\,Z_{R}\,h_{\mathrm{CG}}\,k_{R}\,\cos\left(\mathrm{\phi}\right)\,\sin\left(\mathrm{\theta}\right)-{\mathrm{\dot \psi}}^2\,\mathrm{Iyy}\,\cos\left(\mathrm{\theta}\right)\,\sin\left(\mathrm{\theta}\right)\,{\sin\left(\mathrm{\phi}\right)}^2+2\,Z_{F}\,k_{F}\,z_{\mathrm{CG}}\,\cos\left(\mathrm{\phi}\right)\,\sin\left(\mathrm{\theta}\right)+2\,Z_{R}\,k_{R}\,z_{\mathrm{CG}}\,\cos\left(\mathrm{\phi}\right)\,\sin\left(\mathrm{\theta}\right)+2\,{Z_{F}}^2\,k_{F}\,\cos\left(\mathrm{\theta}\right)\,{\cos\left(\mathrm{\phi}\right)}^2\,\sin\left(\mathrm{\theta}\right)+2\,{Z_{R}}^2\,k_{R}\,\cos\left(\mathrm{\theta}\right)\,{\cos\left(\mathrm{\phi}\right)}^2\,\sin\left(\mathrm{\theta}\right)+\frac{{t_{F}}^2\,k_{F}\,\cos\left(\mathrm{\theta}\right)\,\sin\left(\mathrm{\theta}\right)\,{\sin\left(\mathrm{\phi}\right)}^2}{2}+\frac{{t_{R}}^2\,k_{R}\,\cos\left(\mathrm{\theta}\right)\,\sin\left(\mathrm{\theta}\right)\,{\sin\left(\mathrm{\phi}\right)}^2}{2}+2\,Z_{F}\,k_{F}\,l_{F}\,{\cos\left(\mathrm{\theta}\right)}^2\,\cos\left(\mathrm{\phi}\right)-2\,Z_{R}\,k_{R}\,l_{R}\,{\cos\left(\mathrm{\theta}\right)}^2\,\cos\left(\mathrm{\phi}\right)-2\,Z_{F}\,k_{F}\,l_{F}\,\cos\left(\mathrm{\phi}\right)\,{\sin\left(\mathrm{\theta}\right)}^2+2\,Z_{R}\,k_{R}\,l_{R}\,\cos\left(\mathrm{\phi}\right)\,{\sin\left(\mathrm{\theta}\right)}^2-2\,\mathrm{\dot \phi}\,\mathrm{\dot \psi}\,\mathrm{Ixz}\,\cos\left(\mathrm{\phi}\right)\,\sin\left(\mathrm{\theta}\right)-2\,\mathrm{\dot \phi}\,Z_{F}\,b_{F}\,l_{F}\,{\cos\left(\mathrm{\theta}\right)}^2\,\sin\left(\mathrm{\phi}\right)+2\,\mathrm{\dot \phi}\,Z_{R}\,b_{R}\,l_{R}\,{\cos\left(\mathrm{\theta}\right)}^2\,\sin\left(\mathrm{\phi}\right)+\frac{\mathrm{\dot \phi}\,{t_{F}}^2\,b_{F}\,\cos\left(\mathrm{\theta}\right)\,\cos\left(\mathrm{\phi}\right)\,\sin\left(\mathrm{\theta}\right)\,\sin\left(\mathrm{\phi}\right)}{2}+\frac{\mathrm{\dot \phi}\,{t_{R}}^2\,b_{R}\,\cos\left(\mathrm{\theta}\right)\,\cos\left(\mathrm{\phi}\right)\,\sin\left(\mathrm{\theta}\right)\,\sin\left(\mathrm{\phi}\right)}{2}-2\,\mathrm{\dot \phi}\,{Z_{F}}^2\,b_{F}\,\cos\left(\mathrm{\theta}\right)\,\cos\left(\mathrm{\phi}\right)\,\sin\left(\mathrm{\theta}\right)\,\sin\left(\mathrm{\phi}\right)-2\,\mathrm{\dot \phi}\,{Z_{R}}^2\,b_{R}\,\cos\left(\mathrm{\theta}\right)\,\cos\left(\mathrm{\phi}\right)\,\sin\left(\mathrm{\theta}\right)\,\sin\left(\mathrm{\phi}\right)-4\,\mathrm{\dot \theta}\,Z_{F}\,b_{F}\,l_{F}\,\cos\left(\mathrm{\theta}\right)\,\cos\left(\mathrm{\phi}\right)\,\sin\left(\mathrm{\theta}\right)+4\,\mathrm{\dot \theta}\,Z_{R}\,b_{R}\,l_{R}\,\cos\left(\mathrm{\theta}\right)\,\cos\left(\mathrm{\phi}\right)\,\sin\left(\mathrm{\theta}\right) $

$f_3=    \frac{{\mathrm{\dot \theta}}^2\,\mathrm{Izz}\,\sin\left(2\,\mathrm{\phi}\right)}{2}-\frac{{\mathrm{\dot \theta}}^2\,\mathrm{Iyy}\,\sin\left(2\,\mathrm{\phi}\right)}{2}+\mathrm{\dot \theta}\,\mathrm{\dot \psi}\,\mathrm{Ixx}\,\cos\left(\mathrm{\theta}\right)-2\,{Z_{F}}^2\,k_{F}\,\cos\left(\mathrm{\theta}\right)\,\sin\left(\mathrm{\phi}\right)-2\,{Z_{R}}^2\,k_{R}\,\cos\left(\mathrm{\theta}\right)\,\sin\left(\mathrm{\phi}\right)+2\,\mathrm{\dot z}_{\mathrm{CG}}\,Z_{F}\,b_{F}\,\cos\left(\mathrm{\theta}\right)\,\sin\left(\mathrm{\phi}\right)+2\,\mathrm{\dot z}_{\mathrm{CG}}\,Z_{R}\,b_{R}\,\cos\left(\mathrm{\theta}\right)\,\sin\left(\mathrm{\phi}\right)-\frac{\mathrm{\dot \phi}\,{t_{F}}^2\,b_{F}\,{\cos\left(\mathrm{\theta}\right)}^2\,{\cos\left(\mathrm{\phi}\right)}^2}{2}-\frac{\mathrm{\dot \phi}\,{t_{R}}^2\,b_{R}\,{\cos\left(\mathrm{\theta}\right)}^2\,{\cos\left(\mathrm{\phi}\right)}^2}{2}-2\,\mathrm{\dot \phi}\,{Z_{F}}^2\,b_{F}\,{\cos\left(\mathrm{\theta}\right)}^2\,{\sin\left(\mathrm{\phi}\right)}^2-2\,\mathrm{\dot \phi}\,{Z_{R}}^2\,b_{R}\,{\cos\left(\mathrm{\theta}\right)}^2\,{\sin\left(\mathrm{\phi}\right)}^2+\mathrm{\dot \theta}\,\mathrm{\dot \psi}\,\mathrm{Iyy}\,\cos\left(\mathrm{\theta}\right)\,{\cos\left(\mathrm{\phi}\right)}^2-\mathrm{\dot \theta}\,\mathrm{\dot \psi}\,\mathrm{Izz}\,\cos\left(\mathrm{\theta}\right)\,{\cos\left(\mathrm{\phi}\right)}^2+{\mathrm{\dot \psi}}^2\,\mathrm{Iyy}\,{\cos\left(\mathrm{\theta}\right)}^2\,\cos\left(\mathrm{\phi}\right)\,\sin\left(\mathrm{\phi}\right)-{\mathrm{\dot \psi}}^2\,\mathrm{Izz}\,{\cos\left(\mathrm{\theta}\right)}^2\,\cos\left(\mathrm{\phi}\right)\,\sin\left(\mathrm{\phi}\right)+2\,Z_{F}\,h_{\mathrm{CG}}\,k_{F}\,\cos\left(\mathrm{\theta}\right)\,\sin\left(\mathrm{\phi}\right)+2\,Z_{R}\,h_{\mathrm{CG}}\,k_{R}\,\cos\left(\mathrm{\theta}\right)\,\sin\left(\mathrm{\phi}\right)-\mathrm{\dot \theta}\,\mathrm{\dot \psi}\,\mathrm{Iyy}\,\cos\left(\mathrm{\theta}\right)\,{\sin\left(\mathrm{\phi}\right)}^2+\mathrm{\dot \theta}\,\mathrm{\dot \psi}\,\mathrm{Izz}\,\cos\left(\mathrm{\theta}\right)\,{\sin\left(\mathrm{\phi}\right)}^2+2\,Z_{F}\,k_{F}\,z_{\mathrm{CG}}\,\cos\left(\mathrm{\theta}\right)\,\sin\left(\mathrm{\phi}\right)+2\,Z_{R}\,k_{R}\,z_{\mathrm{CG}}\,\cos\left(\mathrm{\theta}\right)\,\sin\left(\mathrm{\phi}\right)-\frac{{t_{F}}^2\,k_{F}\,{\cos\left(\mathrm{\theta}\right)}^2\,\cos\left(\mathrm{\phi}\right)\,\sin\left(\mathrm{\phi}\right)}{2}-\frac{{t_{R}}^2\,k_{R}\,{\cos\left(\mathrm{\theta}\right)}^2\,\cos\left(\mathrm{\phi}\right)\,\sin\left(\mathrm{\phi}\right)}{2}+2\,{Z_{F}}^2\,k_{F}\,{\cos\left(\mathrm{\theta}\right)}^2\,\cos\left(\mathrm{\phi}\right)\,\sin\left(\mathrm{\phi}\right)+2\,{Z_{R}}^2\,k_{R}\,{\cos\left(\mathrm{\theta}\right)}^2\,\cos\left(\mathrm{\phi}\right)\,\sin\left(\mathrm{\phi}\right)+{\mathrm{\dot \psi}}^2\,\mathrm{Ixz}\,\cos\left(\mathrm{\theta}\right)\,\sin\left(\mathrm{\theta}\right)\,\sin\left(\mathrm{\phi}\right)+2\,\mathrm{\dot \theta}\,\mathrm{\dot \psi}\,\mathrm{Ixz}\,\cos\left(\mathrm{\phi}\right)\,\sin\left(\mathrm{\theta}\right)-2\,Z_{F}\,k_{F}\,l_{F}\,\cos\left(\mathrm{\theta}\right)\,\sin\left(\mathrm{\theta}\right)\,\sin\left(\mathrm{\phi}\right)+2\,Z_{R}\,k_{R}\,l_{R}\,\cos\left(\mathrm{\theta}\right)\,\sin\left(\mathrm{\theta}\right)\,\sin\left(\mathrm{\phi}\right)-2\,\mathrm{\dot \theta}\,Z_{F}\,b_{F}\,l_{F}\,{\cos\left(\mathrm{\theta}\right)}^2\,\sin\left(\mathrm{\phi}\right)+2\,\mathrm{\dot \theta}\,Z_{R}\,b_{R}\,l_{R}\,{\cos\left(\mathrm{\theta}\right)}^2\,\sin\left(\mathrm{\phi}\right)+\frac{\mathrm{\dot \theta}\,{t_{F}}^2\,b_{F}\,\cos\left(\mathrm{\theta}\right)\,\cos\left(\mathrm{\phi}\right)\,\sin\left(\mathrm{\theta}\right)\,\sin\left(\mathrm{\phi}\right)}{2}+\frac{\mathrm{\dot \theta}\,{t_{R}}^2\,b_{R}\,\cos\left(\mathrm{\theta}\right)\,\cos\left(\mathrm{\phi}\right)\,\sin\left(\mathrm{\theta}\right)\,\sin\left(\mathrm{\phi}\right)}{2}-2\,\mathrm{\dot \theta}\,{Z_{F}}^2\,b_{F}\,\cos\left(\mathrm{\theta}\right)\,\cos\left(\mathrm{\phi}\right)\,\sin\left(\mathrm{\theta}\right)\,\sin\left(\mathrm{\phi}\right)-2\,\mathrm{\dot \theta}\,{Z_{R}}^2\,b_{R}\,\cos\left(\mathrm{\theta}\right)\,\cos\left(\mathrm{\phi}\right)\,\sin\left(\mathrm{\theta}\right)\,\sin\left(\mathrm{\phi}\right)$

$f_6=    2\,Z_{F}\,k_{F}-2\,\mathrm{\dot z}_{\mathrm{CG}}\,b_{R}-2\,\mathrm{\dot z}_{\mathrm{CG}}\,b_{F}+2\,Z_{R}\,k_{R}-2\,h_{\mathrm{CG}}\,k_{F}-2\,h_{\mathrm{CG}}\,k_{R}+g\,m-2\,k_{F}\,z_{\mathrm{CG}}-2\,k_{R}\,z_{\mathrm{CG}}+2\,k_{F}\,l_{F}\,\sin\left(\mathrm{\theta}\right)-2\,k_{R}\,l_{R}\,\sin\left(\mathrm{\theta}\right)+2\,\mathrm{\dot \theta}\,b_{F}\,l_{F}\,\cos\left(\mathrm{\theta}\right)-2\,\mathrm{\dot \theta}\,b_{R}\,l_{R}\,\cos\left(\mathrm{\theta}\right)-2\,Z_{F}\,k_{F}\,\cos\left(\mathrm{\theta}\right)\,\cos\left(\mathrm{\phi}\right)-2\,Z_{R}\,k_{R}\,\cos\left(\mathrm{\theta}\right)\,\cos\left(\mathrm{\phi}\right)+2\,\mathrm{\dot \theta}\,Z_{F}\,b_{F}\,\cos\left(\mathrm{\phi}\right)\,\sin\left(\mathrm{\theta}\right)+2\,\mathrm{\dot \theta}\,Z_{R}\,b_{R}\,\cos\left(\mathrm{\phi}\right)\,\sin\left(\mathrm{\theta}\right)+2\,\mathrm{\dot \phi}\,Z_{F}\,b_{F}\,\cos\left(\mathrm{\theta}\right)\,\sin\left(\mathrm{\phi}\right)+2\,\mathrm{\dot \phi}\,Z_{R}\,b_{R}\,\cos\left(\mathrm{\theta}\right)\,\sin\left(\mathrm{\phi}\right)'$
\end{changemargin}

Where $Z_f = h_{CG}-q_f$ and $Z_r = h_{CG}-q_r$.
The inputs vector $u$ is made from the longitudinal and lateral forces acting on each wheel.
$$
u = \left(\begin{array}{c} \mathrm{FX}_{\mathrm{FR}}\\ \mathrm{FY}_{\mathrm{FR}}\\ \mathrm{FX}_{\mathrm{FL}}\\ \mathrm{FY}_{\mathrm{FL}}\\ \mathrm{FX}_{\mathrm{RR}}\\ \mathrm{FY}_{\mathrm{RR}}\\ \mathrm{FX}_{\mathrm{RL}}\\ \mathrm{FY}_{\mathrm{RL}} \end{array}\right)
$$
$$
Q = \quad\quad\quad\quad\quad\quad
$$
$$
\left(\begin{array}{cccccccc} Q_{1,1} & Q_{1,2} & Q_{1,3} & Q_{1,4} & Q_{1,5} & Q_{1,6} & Q_{1,7} & Q_{1,8}\\ Q_{2,1} & Q_{2,2} & Q_{2,3} & Q_{2,4} & Q_{2,5} & Q_{2,6} & Q_{2,7} & Q_{2,8}\\ Q_{3,1} & Q_{3,2} & Q_{3,3} & Q_{3,4} & Q_{3,5} & Q_{3,6} & Q_{3,7} & Q_{3,8}\\ \cos\left(\mathrm{\delta}+\mathrm{\psi}\right) & -\sin\left(\mathrm{\delta}+\mathrm{\psi}\right) & \cos\left(\mathrm{\delta}+\mathrm{\psi}\right) & -\sin\left(\mathrm{\delta}+\mathrm{\psi}\right) & \cos\left(\mathrm{\psi}\right) & -\sin\left(\mathrm{\psi}\right) & \cos\left(\mathrm{\psi}\right) & -\sin\left(\mathrm{\psi}\right)\\ \sin\left(\mathrm{\delta}+\mathrm{\psi}\right) & \cos\left(\mathrm{\delta}+\mathrm{\psi}\right) & \sin\left(\mathrm{\delta}+\mathrm{\psi}\right) & \cos\left(\mathrm{\delta}+\mathrm{\psi}\right) & \sin\left(\mathrm{\psi}\right) & \cos\left(\mathrm{\psi}\right) & \sin\left(\mathrm{\psi}\right) & \cos\left(\mathrm{\psi}\right)\\ 0 & 0 & 0 & 0 & 0 & 0 & 0 & 0 \end{array}\right)
$$

\begin{align*}
Q_{1,1}&=Z_{f}\,\cos\left(\mathrm{\delta}\right)\,\sin\left(\mathrm{\phi}\right)-\frac{t_{f}\,\cos\left(\mathrm{\phi}\right)\,\cos\left(\mathrm{\delta}\right)}{2}+l_{f}\,\cos\left(\mathrm{\theta}\right)\,\sin\left(\mathrm{\delta}\right)+\\&+Z_{f}\,\cos\left(\mathrm{\phi}\right)\,\sin\left(\mathrm{\theta}\right)\,\sin\left(\mathrm{\delta}\right)+\frac{t_{f}\,\sin\left(\mathrm{\theta}\right)\,\sin\left(\mathrm{\phi}\right)\,\sin\left(\mathrm{\delta}\right)}{2}\\
Q_{2,1}&=\frac{\cos\left(\mathrm{\delta}\right)\,\left(2\,Z_{f}\,\cos\left(\mathrm{\theta}\right)\,\cos\left(\mathrm{\phi}\right)-2\,l_{f}\,\sin\left(\mathrm{\theta}\right)+T_{f}\,\cos\left(\mathrm{\theta}\right)\,\sin\left(\mathrm{\phi}\right)\right)}{2}\\
Q_{3,1}&=\frac{t_{f}\,\cos\left(\mathrm{\phi}\right)\,\cos\left(\mathrm{\delta}\right)\,\sin\left(\mathrm{\theta}\right)}{2}-\frac{t_{f}\,\sin\left(\mathrm{\phi}\right)\,\sin\left(\mathrm{\delta}\right)}{2}+\\&-Z_{f}\,\cos\left(\mathrm{\phi}\right)\,\sin\left(\mathrm{\delta}\right)-Z_{f}\,\cos\left(\mathrm{\delta}\right)\,\sin\left(\mathrm{\theta}\right)\,\sin\left(\mathrm{\phi}\right)\\
Q_{1,2}&=\frac{t_{f}\,\cos\left(\mathrm{\phi}\right)\,\sin\left(\mathrm{\delta}\right)}{2}+l_{f}\,\cos\left(\mathrm{\theta}\right)\,\cos\left(\mathrm{\delta}\right)-Z_{f}\,\sin\left(\mathrm{\phi}\right)\,\sin\left(\mathrm{\delta}\right)+\\&+Z_{f}\,\cos\left(\mathrm{\phi}\right)\,\cos\left(\mathrm{\delta}\right)\,\sin\left(\mathrm{\theta}\right)+\frac{t_{f}\,\cos\left(\mathrm{\delta}\right)\,\sin\left(\mathrm{\theta}\right)\,\sin\left(\mathrm{\phi}\right)}{2}\\
Q_{2,2}&=-\frac{\sin\left(\mathrm{\delta}\right)\,\left(2\,Z_{f}\,\cos\left(\mathrm{\theta}\right)\,\cos\left(\mathrm{\phi}\right)-2\,l_{f}\,\sin\left(\mathrm{\theta}\right)+T_{f}\,\cos\left(\mathrm{\theta}\right)\,\sin\left(\mathrm{\phi}\right)\right)}{2}\\
Q_{3,2}&=Z_{f}\,\sin\left(\mathrm{\theta}\right)\,\sin\left(\mathrm{\phi}\right)\,\sin\left(\mathrm{\delta}\right)-\frac{t_{f}\,\cos\left(\mathrm{\delta}\right)\,\sin\left(\mathrm{\phi}\right)}{2}-\frac{t_{f}\,\cos\left(\mathrm{\phi}\right)\,\sin\left(\mathrm{\theta}\right)\,\sin\left(\mathrm{\delta}\right)}{2}+\\&-Z_{f}\,\cos\left(\mathrm{\phi}\right)\,\cos\left(\mathrm{\delta}\right)\\
Q_{1,3}&=\frac{t_{f}\,\cos\left(\mathrm{\phi}\right)\,\cos\left(\mathrm{\delta}\right)}{2}+Z_{f}\,\cos\left(\mathrm{\delta}\right)\,\sin\left(\mathrm{\phi}\right)+l_{f}\,\cos\left(\mathrm{\theta}\right)\,\sin\left(\mathrm{\delta}\right)+\\&+Z_{f}\,\cos\left(\mathrm{\phi}\right)\,\sin\left(\mathrm{\theta}\right)\,\sin\left(\mathrm{\delta}\right)-\frac{t_{f}\,\sin\left(\mathrm{\theta}\right)\,\sin\left(\mathrm{\phi}\right)\,\sin\left(\mathrm{\delta}\right)}{2}\\
Q_{2,3}&=-\frac{\cos\left(\mathrm{\delta}\right)\,\left(2\,l_{f}\,\sin\left(\mathrm{\theta}\right)-2\,Z_{f}\,\cos\left(\mathrm{\theta}\right)\,\cos\left(\mathrm{\phi}\right)+T_{f}\,\cos\left(\mathrm{\theta}\right)\,\sin\left(\mathrm{\phi}\right)\right)}{2}\\
Q_{3,3}&=\frac{t_{f}\,\sin\left(\mathrm{\phi}\right)\,\sin\left(\mathrm{\delta}\right)}{2}-Z_{f}\,\cos\left(\mathrm{\phi}\right)\,\sin\left(\mathrm{\delta}\right)-\frac{t_{f}\,\cos\left(\mathrm{\phi}\right)\,\cos\left(\mathrm{\delta}\right)\,\sin\left(\mathrm{\theta}\right)}{2}+\\&-Z_{f}\,\cos\left(\mathrm{\delta}\right)\,\sin\left(\mathrm{\theta}\right)\,\sin\left(\mathrm{\phi}\right)\\
Q_{1,4}&=l_{f}\,\cos\left(\mathrm{\theta}\right)\,\cos\left(\mathrm{\delta}\right)-\frac{t_{f}\,\cos\left(\mathrm{\phi}\right)\,\sin\left(\mathrm{\delta}\right)}{2}-Z_{f}\,\sin\left(\mathrm{\phi}\right)\,\sin\left(\mathrm{\delta}\right)+\\&+Z_{f}\,\cos\left(\mathrm{\phi}\right)\,\cos\left(\mathrm{\delta}\right)\,\sin\left(\mathrm{\theta}\right)-\frac{t_{f}\,\cos\left(\mathrm{\delta}\right)\,\sin\left(\mathrm{\theta}\right)\,\sin\left(\mathrm{\phi}\right)}{2}\\
Q_{2,4}&=\frac{\sin\left(\mathrm{\delta}\right)\,\left(2\,l_{f}\,\sin\left(\mathrm{\theta}\right)-2\,Z_{f}\,\cos\left(\mathrm{\theta}\right)\,\cos\left(\mathrm{\phi}\right)+T_{f}\,\cos\left(\mathrm{\theta}\right)\,\sin\left(\mathrm{\phi}\right)\right)}{2}\\
Q_{3,4}&=\frac{t_{f}\,\cos\left(\mathrm{\delta}\right)\,\sin\left(\mathrm{\phi}\right)}{2}-Z_{f}\,\cos\left(\mathrm{\phi}\right)\,\cos\left(\mathrm{\delta}\right)+\frac{t_{f}\,\cos\left(\mathrm{\phi}\right)\,\sin\left(\mathrm{\theta}\right)\,\sin\left(\mathrm{\delta}\right)}{2}+\\&+Z_{f}\,\sin\left(\mathrm{\theta}\right)\,\sin\left(\mathrm{\phi}\right)\,\sin\left(\mathrm{\delta}\right)\\
Q_{1,5}&=Z_{r}\,\sin\left(\mathrm{\phi}\right)-\frac{t_{r}\,\cos\left(\mathrm{\phi}\right)}{2}\\
Q_{2,5}&=l_{r}\,\sin\left(\mathrm{\theta}\right)+Z_{r}\,\cos\left(\mathrm{\theta}\right)\,\cos\left(\mathrm{\phi}\right)+\frac{t_{r}\,\cos\left(\mathrm{\theta}\right)\,\sin\left(\mathrm{\phi}\right)}{2}\\
Q_{3,5}&=\frac{\sin\left(\mathrm{\theta}\right)\,\left(T_{r}\,\cos\left(\mathrm{\phi}\right)-2\,Z_{r}\,\sin\left(\mathrm{\phi}\right)\right)}{2}\\
\end{align*}
\begin{align*}
Q_{1,6}&=Z_{r}\,\cos\left(\mathrm{\phi}\right)\,\sin\left(\mathrm{\theta}\right)-l_{r}\,\cos\left(\mathrm{\theta}\right)+\frac{t_{r}\,\sin\left(\mathrm{\theta}\right)\,\sin\left(\mathrm{\phi}\right)}{2}\\
Q_{2,6}&=0\\
Q_{3,6}&=-Z_{r}\,\cos\left(\mathrm{\phi}\right)-\frac{t_{r}\,\sin\left(\mathrm{\phi}\right)}{2}\\
Q_{1,7}&=\frac{t_{r}\,\cos\left(\mathrm{\phi}\right)}{2}+Z_{r}\,\sin\left(\mathrm{\phi}\right)\\
Q_{2,7}&=l_{r}\,\sin\left(\mathrm{\theta}\right)+Z_{r}\,\cos\left(\mathrm{\theta}\right)\,\cos\left(\mathrm{\phi}\right)-\frac{t_{r}\,\cos\left(\mathrm{\theta}\right)\,\sin\left(\mathrm{\phi}\right)}{2}\\
Q_{3,7}&=-\frac{\sin\left(\mathrm{\theta}\right)\,\left(T_{r}\,\cos\left(\mathrm{\phi}\right)+2\,Z_{r}\,\sin\left(\mathrm{\phi}\right)\right)}{2}\\
Q_{1,8}&=Z_{r}\,\cos\left(\mathrm{\phi}\right)\,\sin\left(\mathrm{\theta}\right)-l_{r}\,\cos\left(\mathrm{\theta}\right)-\frac{t_{r}\,\sin\left(\mathrm{\theta}\right)\,\sin\left(\mathrm{\phi}\right)}{2}\\
Q_{2,8}&=0\\
Q_{3,8}&=\frac{t_{r}\,\sin\left(\mathrm{\phi}\right)}{2}-Z_{r}\,\cos\left(\mathrm{\phi}\right)\\
\end{align*}


\section{Output functions}
\label{sec:6dofout}
The vertical wheel load outputs are calculated by the forces acting in the suspension systems, as a sum of the elastic and linear damping force.
$$Z_{FR} = - k_F (h_{FR} - h_{F0}) - b_F \dot h_{FR} $$
$$Z_{FL} = - k_F (h_{FL} - h_{F0}) - b_F \dot h_{FL} $$
$$Z_{RR} = - k_R (h_{RR} - h_{R0}) - b_R \dot h_{RR} $$
$$Z_{RL} = - k_R (h_{RL} - h_{R0}) - b_R \dot h_{RL} $$

The contact point velocities output is given by the time derivatives of the $W_w$ points' co-ordinates in the inertial frame rotated into wheel system co-ordinates. The $W_w$ point coordinates with respect to the undercarriage reference frame are defined in table \ref{table:contactpoints}.

\begin{table}[ht]
  \caption{Coordinates of the tyre contact points} % title of Table
  \centering % used for centering table
  \begin{tabular}{l l l l l} % left aligned columns (4 columns)
    \hline\hline %inserts double horizontal lines
    Undercarriage system co-ordinates & $W_{FR}$ & $W_{FL}$ & $W_{RR}$ & $W_{RL}$ \\ [0.5ex] % inserts table
    %heading
    \hline % inserts single horizontal line
    $ x$ & $ l_f$ & $ l_f$ & $-l_r $ & $-l_r $\\ % inserting body of the table
    $ y$ & $ \frac{t_f}{2} $ & $ -\frac{t_f}{2}$ & $ \frac{t_r}{2}$ & $ -\frac{t_r}{2}$\\ % inserting body of the table
    $ z$ & $ 0 $& $ 0 $ & $ 0$ & $ 0$ \\ [1ex] % [1ex] adds vertical space
    \hline %inserts single line
  \end{tabular}
  \label{table:contactpoints} % is used to refer this table in the text
\end{table}
